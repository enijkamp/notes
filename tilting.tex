\documentclass[11pt]{article}

\usepackage{geometry}
\geometry{letterpaper}                   
\usepackage{graphicx}
\usepackage{amssymb}
\usepackage{epstopdf}
\usepackage{amsmath}
\usepackage{filecontents}

\begin{filecontents}{\jobname.bib}
\end{filecontents}

\def\E{\mathcal{E}}
\def\F{\mathcal{F}}

\begin{document}
	
\begin{center}
{\LARGE On Exponential Tilting} 
\end{center}

In Monte Carlo estimation, exponential tilting is a distribution shifting technique used in rare-event simulation, rejection and importance sampling.

\section{Exponential Tilting}

Given a random variable $X$ with probability distribution $P$, density $f$, and m.g.f. $M_X(\theta) = E[e^{\theta X}] < \infty$, the exponentially titled measure $P_\theta$ is:

\begin{equation}
P_\theta(X \in dx) = \frac{E[e^{\theta X}1[X\in dx]]}{M_X (\theta)} = e^{\theta x - \kappa(\theta)} P(X \in dx)
\end{equation}

where $\kappa(\theta)$ is the c.g.f. $\log E[e^{\theta X}]$. Then $P_\theta$ has density

\begin{equation}
f_\theta(x) = \frac{e^{\theta x}f(x)}{M_X(\theta)}=\exp\{\theta x - \kappa(\theta) \}f(x).
\end{equation}

\section{Esscher Transform}

Let $f(x)$ be a probability density. Then its Esscher transform is:

\begin{equation}
f_\theta(x) = \frac{ e^{\theta x}f(x) }{ \int_{-\infty}^{\infty} e^{\theta x}f(x)dx }
\end{equation}

The effect of the Esscher transform on the normal distribution is a mean shifting:

\begin{equation}
E_\theta(\mathcal{N}(\mu,\sigma^2)) = \mathcal{N}(\mu+\theta\sigma^2,\sigma^2).
\end{equation}

\section{Gibbs and ReLU}

Consider the distribution with parameters $\theta$:

\begin{eqnarray}
p_\theta(X) = \frac{1}{Z(\theta)} \exp\left[ f_\theta(X)\right] p_0(X)
\label{eq:gibbs}
\end{eqnarray}

where $p_0$ is Gaussian noise as reference distribution and $Z(\theta)$ denotes the partition function

\begin{eqnarray}
Z(\theta) = \int\exp[f_\theta(X)]p_0(X)dX.
\end{eqnarray}

Then $p_\theta(Y)$ is an exponential tilting of $p_0$. In the trivial case of $f_\theta(X)=\theta X$, equation (\ref{eq:gibbs}) is a Gibbs distribution and the effect of exponential tilting on $p_0(Y)\sim\mathcal{N}(\mu,\sigma^2)$ reduces to mean shifting $p_\theta(X)\sim\mathcal{N}(\mu+\theta \sigma^2,\sigma^2)$. If $f_\theta(X)$ is a ConvNet with piecewise linear rectification, then $f_\theta(X)$ is piecewise linear in $X$, and $p_\theta$ is piecewise mean shifted Gaussian. 

\bibliographystyle{plain}
\bibliography{\jobname} 

\end{document}
